\documentclass[11pt]{article}
\usepackage{graphicx}
\graphicspath{ {./img/} }
\begin{document}

\title{ Clasificaci\'on de flores usando el dataset Iris por medio de una red neuronal }
\author{Orlando Hernandez Nu\~nez. Juan Alejandro Bernal Gallego}
\date{05 de Noviembre del 2019}
\maketitle
\section{Introducci\'on}
En el presente trabajo se describe el proceso llevado a cabo para entrenar una red neuronal con el objetivo de clasificar una flor atendiendo a sus caracter\'isticas f\'isicas.
\section{Red neuronal}
\subsection{Descripci\'on del dataset}
El dataset con el que se realiza el entrenamiento de la red neuronal se compone de 100 registros o muestras,
cada uno de ellos con la medida de longitud y ancho de p\'etalos y s\'epalos adem\'as de su respectiva clasificaci\'on.
\subsection{Arquitectura de la red neuronal}
La red neuronal posee tres capas, una capa de entrada donde se tomar\'an 4 valores, que corresponden a las caracter\'isticas físicas del tama\~no de una flor; una capa intermedia oculta la cual
posee 1 2 \'o 3 neuronas y finalmente, una capa de salida que esta compuesta de una neurona.
\begin{figure}[h]
    \includegraphics[width=10cm, height=8cm]{1hn}
    \centering
    \caption{Red neuronal con 1 neurona en la capa oculta}
    \label{1hn}
\end{figure}

\begin{figure}[h]
    \includegraphics[width=10cm, height=8cm]{2hn}
    \centering
    \caption{Red neuronal con 2 neuronas en la capa oculta}
    \label{2hn}
\end{figure}
\begin{figure}[h]
    \includegraphics[width=10cm, height=8cm]{3hn}
    \centering
    \caption{Red neuronal con 3 neuronas en la capa oculta}
    \label{3hn}
\end{figure}
\subsection{Entrenamiento}
De manera aleatoria se escoge  el 80\% del dataset para entrenar la red neuronal y el porcentaje
restante es usado para evaluar los resultados obtenidos una vez hecho el entrenamiento.
El entrenamiento se realiz\'o con cada uno de los dise\~nos mostrados en \ref{1hn}, \ref{2hn} y \ref{3hn}.
Como par\'ametros de entrada para el proceso de entrenamiento se toma el eta o taza de aprendizaje, el n\'umero de epocas a alcanzar y la funci\'on de activaci\'on.
Los pesos al inicio del proceso de entrenamiento son generados de manera aleatoria. Las funciones de activaci\'on disponibles son...
\subsubsection{Entrenamiento con una neurona en la capa oculta}
Con eta=4.5 y hasta alcanzar la \'epoca 1000 se obtuvo:
\subsubsection{Entrenamiento con dos neuronas en la capa oculta}
Con eta=4.5 y hasta alcanzar la \'epoca 1000 se obtuvo:
\subsubsection{Entrenamiento con tres neuronas en la capa oculta}
Con eta=4.5 y hasta alcanzar la \'epoca 1000 se obtuvo:
\end{document}
