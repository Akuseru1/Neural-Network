% Created 2019-11-05 Tue 15:32
% Intended LaTeX compiler: pdflatex
\documentclass[11pt]{article}
\usepackage[utf8]{inputenc}
\usepackage[T1]{fontenc}
\usepackage{graphicx}
\usepackage{grffile}
\usepackage{longtable}
\usepackage{wrapfig}
\usepackage{rotating}
\usepackage[normalem]{ulem}
\usepackage{amsmath}
\usepackage{textcomp}
\usepackage{amssymb}
\usepackage{capt-of}
\usepackage{hyperref}
\date{\today}
\title{}
\hypersetup{
 pdfauthor={},
 pdftitle={},
 pdfkeywords={},
 pdfsubject={},
 pdfcreator={Emacs 26.3 (Org mode 9.1.9)},
 pdflang={English}}
\begin{document}

\tableofcontents

\section{Prediccion de clase de Iris Virginica e Iris Setosa haciendo uso de una red neuronal}
\label{sec:orgaf12cb4}

\subsection{Introducción}
\label{sec:org6d27da3}
El objetivo de este trabajo es predecir correctamente la clase de un subgrupo (el 20\%) de flores iris con la menor cantidad de error posible, para esto se utilizo una red neuronal la cual fue entrenada con el 80\% de los datos. A continuación mostraremos los parámetros utilizados para conseguir un error de 10\(^{\text{-3}}\)

\subsection{Descripción de la base de datos}
\label{sec:org652ef46}
La base de datos Iris contiene la variación morfológica de las 3 especies de flor, Iris Setosa, Iris Virginica e Iris Versicolor. La base datos de datos consiste de una colección de 50 flores y contiene información sobre 4 características:
\begin{equation}
Anchura del pétalo
\end{equation}
\begin{equation}
Anchura del sépalo
\end{equation}
\begin{equation}
Altura del pétalo
\end{equation}
\begin{equation}
Altura del sépalo
\end{equation}

\subsection{Red Neuronal}
\label{sec:orgec229dc}
\subsubsection{Función}
\label{sec:org6d11709}
La red neuronal permite el uso de
\end{document}
